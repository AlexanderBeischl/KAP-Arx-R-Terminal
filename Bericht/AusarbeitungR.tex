%%%%%%%%%%%%%%%%%%%%%%%%%%%%%%%%%%%%%%%%%
% Thin Sectioned Essay
% LaTeX Template
% Version 1.0 (3/8/13)
%
% This template is based on a downloaded from:
% http://www.LaTeXTemplates.com
%
% Original Author:
% Nicolas Diaz (nsdiaz@uc.cl) with extensive modifications by:
% Vel (vel@latextemplates.com)
% and adjustments by 
% Thuy Tran 
%
% License:
% CC BY-NC-SA 3.0 (http://creativecommons.org/licenses/by-nc-sa/3.0/)
%
%%%%%%%%%%%%%%%%%%%%%%%%%%%%%%%%%%%%%%%%%

%----------------------------------------------------------------------------------------
%	PACKAGES AND OTHER DOCUMENT CONFIGURATIONS
%----------------------------------------------------------------------------------------

\documentclass[a4paper, 12pt]{report} % Font size (can be 10pt, 11pt or 12pt) and paper size (remove a4paper for US letter paper)

\usepackage[protrusion=true,expansion=true]{microtype} % Better typography
\usepackage{graphicx} % Required for including pictures
\usepackage{wrapfig} % Allows in-line images

\usepackage{mathpazo} % Use the Palatino font
\usepackage[T1]{fontenc} % Required for accented characters
\usepackage[utf8]{inputenc} 
\usepackage[ngerman]{babel}
\linespread{1.05} % Change line spacing here, Palatino benefits from a slight increase by default

\makeatletter
\renewcommand\@biblabel[1]{\textbf{#1.}} % Change the square brackets for each bibliography item from '[1]' to '1.'
\renewcommand{\@listI}{\itemsep=0pt} % Reduce the space between items in the itemize and enumerate environments and the bibliography

\renewcommand{\maketitle}{ % Customize the title - do not edit title and author name here, see the TITLE block below
\begin{flushright} % Right align
{\LARGE\@title} % Increase the font size of the title

\vspace{50pt} % Some vertical space between the title and author name

{\large\@author} % Author name
\\\@date % Date

\vspace{40pt} % Some vertical space between the author block and abstract
\end{flushright}
}

%----------------------------------------------------------------------------------------
%	TITLE
%----------------------------------------------------------------------------------------

\title{\textbf{}\\ % Title
Terminal zur Bearbeitung von ARX-Daten mit R} % Subtitle

\author{\textsc{Alexander Beischl und Thuy Tran} % Author
\\{\textit{Technische Universität München}}} % Institution

\date{\today} % Date

%----------------------------------------------------------------------------------------

\begin{document}

\maketitle % Print the title section

%----------------------------------------------------------------------------------------
%	ABSTRACT AND KEYWORDS
%----------------------------------------------------------------------------------------

\renewcommand{\abstractname}{Zusammenfassung} % Uncomment to change the name of the abstract to something else

\begin{abstract}
Morbi tempor congue porta. Proin semper, leo vitae faucibus dictum, metus mauris lacinia lorem, ac congue leo felis eu turpis. Sed nec nunc pellentesque, gravida eros at, porttitor ipsum. Praesent consequat urna a lacus lobortis ultrices eget ac metus. In tempus hendrerit rhoncus. Mauris dignissim turpis id sollicitudin lacinia. Praesent libero tellus, fringilla nec ullamcorper at, ultrices id nulla. Phasellus placerat a tellus a malesuada.
\end{abstract}


\vspace{30pt} % Some vertical space between the abstract and first section

%----------------------------------------------------------------------------------------
%	ESSAY BODY
%----------------------------------------------------------------------------------------

\chapter{Einführung}\label{einführung}

This statement requires citation \cite{latexcompanion}; this one does too \cite{einstein}. Lorem ipsum dolor sit amet, consectetur adipiscing elit. Aenean dictum lacus sem, ut varius ante dignissim ac. Sed a mi quis lectus feugiat aliquam. Nunc sed vulputate velit. Sed commodo metus vel felis semper, quis rutrum odio vulputate. Donec a elit porttitor, facilisis nisl sit amet, dignissim arcu. Vivamus accumsan pellentesque nulla at euismod. Duis porta rutrum sem, eu facilisis mi varius sed. Suspendisse potenti. Mauris rhoncus neque nisi, ut laoreet augue pretium luctus. Vestibulum sit amet luctus sem, luctus ultrices leo. Aenean vitae sem leo.

Nullam semper quam at ante convallis posuere. Ut faucibus tellus ac massa luctus consectetur. Nulla pellentesque tortor et aliquam vehicula. Maecenas imperdiet euismod enim ut pharetra. Suspendisse pulvinar sapien vitae placerat pellentesque. Nulla facilisi. Aenean vitae nunc venenatis, vehicula neque in, congue ligula.

%------------------------------------------------

\section*{R}\label{r}
R ist eine Programmiersprache und Entwicklungsumgebung, die für statistische Berechnungen und Graphen unter John Chambers von Bell Laboratories entwickelt wurde. Sie ist ein GNU-Projekt, bei dem die Entwicklung von freier Software im Mittelpunkt steht. R weist große Ähnlichkeiten zu der Programmiersprache S auf, ein weiteres GNU-Projekt, welches weitgehend auch unter R läuft. \cite{rproject}

R bietet standardmäßig alle Hauptfunktionen an für die statistische Analyse von Datensätzen an und ist einfach zu erweitern, weswegen es vor allem für wissenschaftliche Arbeiten verwendet wird. Es ist mit R einfach, statistische Funktionen auf große Datenmengen anzuwenden.  


%------------------------------------------------

\section*{ARX}
ARX ist eine freie Software zur Anonymisierung von medizinischen Datensätzen, die von Fabian Prasser und Florian Kohlmayer vom Institut für medizinische Statistik und Epidemiologie an der Technischen Universität München entwickelt wurde.



\chapter{Anwenderdokumentation}
\section{Übersicht}
Das R-Terminal dient dazu, über eine externe Schnittstelle R aufzurufen und zu bedienen. Dies soll vor allem dazu genutzt werden, um Tabellen aus ARX einzuladen und Skripte auszuführen, und gleichzeitig die im Kapitel \ref{einführung} beschriebenen Probleme zu umgehen. Das R-Terminal ist kompatibel mit Windows, der Linux Distribution Ubuntu und OS X, von denen jeweils die folgenden Versionen im Rahmen der Entwicklung getestet wurden: 

\begin{itemize}
\item Windows 10 Education (Version 1511)
\item OS X El Capitan (Version 10.11.1)
\item macOS Sierra (Version 10.12.2)
\item Ubuntu
\end{itemize}

%insert screenshot for linux and the correct ubuntu/opensuse version

In den folgenden Abschnitten werden die Grundfunktionen und zusätzliche Features des R-Terminals beschrieben. 
\section{}

\begin{figure}[htpb]
\centering
\includegraphics[width=0.8\textwidth]{R-TerminalWindows}
\caption{R-Terminal: \textit{Terminal} unter Windows 10 Education (Version 1511)}
\label{rterminalwindows}
\end{figure}

\begin{figure}[htpb]
\centering
\includegraphics[width=0.8\textwidth]{rterminalwindows}
\caption{R-Terminal: \textit{Setup} unter Windows 10 Education (Version 1511)}
\label{rterminalwindows}
\end{figure}

\begin{figure}[htpb]
\centering
\includegraphics[width=0.8\textwidth]{R-Terminal}
\caption{R-Terminal: \textit{Terminal} unter OS X (Version 10.11.1)}
\label{rterminalmac}
\end{figure}

\begin{figure}[htpb]
\centering
\includegraphics[width=0.8\textwidth]{rterminalsetup}
\caption{R-Terminal: \textit{Setup} unter OS X (Version 10.11.1)}
\label{macsetup}
\end{figure}

In Abbildung \ref{rterminalmac} ist das R-Terminal unter OS X (hier Version 10.11.1) zu sehen. Das Terminal verfügt über die beiden Tabs \textit{Terminal} und \textit{Setup} (s. \ref{macsetup)}. Unter \textit{Setup} wird entweder eine R-Version auf dem Rechner gesucht und automatisch ausgeführt oder es wird vom Benutzer selber der Pfad zu der gewünschten R-Version angegeben.

\newpage

\chapter{Entwicklerdokumentation}

\section{Verwendete Technologien}

Das \textit{R-Terminal} wurde mit der Entwicklungsumgebung \textit{Eclipse} entwickelt.
Als Programmiersprache wurde \textit{Java} gewählt. Zur Realisierung der graphischen Benutzeroberfläche wurde \textit{SWT} gewählt, genauere Details hierzu werden in \ref{swt} erläutert.



\section{Architektur des R-Terminals}

Die Software des \textit{R-Terminal} ist in zwei Komponenten aufgeteilt, welche als getrennte "`Packages"' vorliegen. Das erste Package beinhaltet die graphische Benutzeroberfläche, das Zweite beinhaltet die gesamte Integration von \textit{R} in \textit{Java}, es stellt also alle Funktionen zur Verfügung und verwaltet die Kommunikation zwischen \textit{R} und dem \textit{Java}-Programm.

Durch die Trennung von graphischer Benutzeroberfläche und der \textit{R}-Integration können alle Funktionen auch ohne die GUI ausgeführt und verwendet werden.

\section{Graphische Benutzeroberfläche}

\begin{samepage}

Das erste Package, benannt "`org.deidentifier.arx.gui"', erzeugt und verwaltet die graphische Benutzeroberfläche. Es umfasst sechs Klassen:
\begin{itemize}
	\item RMain
	\item RTerminal
	\item RSetupTab
	\item RTerminalTab
	\item RLayout
	\item RCommandListener
\end{itemize}
\end{samepage}

\subsection{RMain}

Um das Programm mit der graphischen Benutzeroberfläche zu starten, muss die Methode "`main"' der Klasse \textit{RMain} aufgerufen werden. Diese erzeugt ein neues Display sowie eine Shell und erzeugt im Anschluss ein neues Objekt der Klasse \textit{RTerminal}.
Letzteres erzeugt die einzelnen Komponenten der Oberfläche.

\subsection{RTerminal}

Die Erzeugung der GUI wird durch den Konstruktor des neuen \textit{RTerminal}-Objektes realisiert. Dieser erzeugt einen TabFolder mit zwei Tabs, welche mit den Klassen \textit{RTerminalTab} und \textit{RSetupTab} befüllt werden.
Außerdem erzeugt der Konstruktor noch einen Ring-Puffer, welcher zur Speicherung des Std-Output von \textit{R} verwendet wird, sowie einen Listener. Sowohl der Ring-Puffer, als auch der Listener wurden im zweiten Package implementiert.
Im Anschluss wird die Integration mit \textit{R} durch den Aufruf der Methode "`startRIntergration"' eingeleitet.\\

Diese Methode erzeugt ein neues Objekt der Klasse "`RIntegration"', welches in \ref{RIntegration} beschrieben wird und die Integration von R realisiert.

\subsection{RSetupTab}

Die Klasse \textit{RSetupTab} erzeugt einen Tab, welcher Informationen zum verwendeten Betriebssystem sowie dem Status der R-Integration anzeigt.\\

Die \textit{SWT}-Labels, welche die Informationen beinhalten, werden durch mehrere Methodenaufrufe im Konstruktor erzeugt.
Das aktuell verwendete Betriebssystem wird durch den Aufruf der Methode "`printOS()"' der Klasse \textit{OS} aus dem zweiten Package aufgerufen. Weitere Informationen hierzu finden sich in \ref{OS}.

Die Integration von \textit{R} wird in der Klasse \textit{RIntegration} des zweiten Package durchgeführt. Diese wird beim Start des Programms aufgerufen und durchsucht die Standart-Installationspfade von \textit{R} nach einer ausführbaren R-Executive.
 Um den Status der Integration zu prüfen, wird aus \textit{OS} die Methode "`getR()"' aufgerufen. Diese gibt den absoluten Pfad zur aktuell verwendeten R-Executive zurück, welcher im \textit{RSetupTab} angezeigt wird.
Wurde keine R-Executive gefunden oder konnte diese nicht erfolgreich gestartet werden, so wird von "`getR()"' \textit{null} zurückgegeben und es wird im \textit{RSetupTab} ausgegeben:  
\begin{center}
Location: "`No falid R-Version selected!"'

Version: "`No falid R-Version selected!"'
\end{center}

Wurde eine R-Version gefunden und erfolgreich ausgeführt, so wird der erzeugte Listener ausgelöst. Dieser ruft anschließend die Methode "`update()"' auf, durch welche die beiden Felder Location und Version aktualisiert werden.\\

Außerdem ermöglicht der \textit{RSetupTab} die manuelle Suche einer \textit{R}-Executive. Hierfür kann entweder der absolute Pfad zur Datei angegeben werden oder diese mittels eines Navigationsfensters des Standart-Dateimanagers ausgewählt werden.
Der RSetupTab beinhaltet hierfür eine Kommandozeile zur Pfadeingabe bzw. einen Knopf, durch welchen das Navigationsfensters aufgerufen wird.

\section{SWT}\label{swt} 
Die graphische Benutzeroberfläche wurde mithilfe des Standard Widget Toolkit (SWT) realisiert. SWT ist ein "open-source widget toolkit" für Java mit Integration in die GUI des nativen Betriebssystems, sodass das Design der Benutzeroberfläche an das Design des Betriebssystems angepasst wird. 

Für das Projekt \textit{R-Terminal} wurde die SWT-Version 4.2.1 verwendet.







%----------------------------------------------------------------------------------------
%	BIBLIOGRAPHY
%----------------------------------------------------------------------------------------

%\bibliographystyle{unsrt}

%\bibliography{sample}
\begin{thebibliography}{9} %Das ist nur ein Beispiel, 1 und 2 sind im dummy text oben eingebunden 
\bibitem{latexcompanion} 
Michel Goossens, Frank Mittelbach, and Alexander Samarin. 
\textit{The \LaTeX\ Companion}. 
Addison-Wesley, Reading, Massachusetts, 1993.
 
\bibitem{einstein} 
Albert Einstein. 
\textit{Zur Elektrodynamik bewegter K{\"o}rper}. (German) 
[\textit{On the electrodynamics of moving bodies}]. 
Annalen der Physik, 322(10):891–921, 1905.

 
\bibitem{rproject}
\texttt{https://www.r-project.org/about.html}
\end{thebibliography}
%----------------------------------------------------------------------------------------

\end{document}